version https://git-lfs.github.com/spec/v1
oid sha256:e7de33c26d6873be8bbe905caf2240e8e9c9aa2a61f872db353713821712133e
size 48264
